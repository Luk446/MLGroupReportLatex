\section{Discussion}

The stroke prediction set revealed important insights into model performance. When the models were applied to the unbalanced data, the results showed high accuracy, but barely predicted any actual strokes (low recall).Which means they usually just guessed “no stroke” every time. Logistic Regression performed best overall as shown in Fig.~\ref{fig:BCM} and table~\ref{tab:stroke_classification_report}, it has a strong recall score of 0.79 for the stroke cases while maintaining a good accuracy. This improvement shows how important it is to deal with unbalanced data when making predictions. The neural network achieved a high accuracy of 92\% but struggled to identify actual stroke cases (low recall), which suggests it had difficulty in learning the patterns of the minority class.

The CNN model demonstrated strong performance in classifying brain tumor types. As shown in the confusion matrix (Fig.~\ref{fig:MRICM}), the model correctly identified the majority of cases for each class, with particularly high accuracy for Brain Glioma (284/301) and Brain Meningioma (274/301). Misclassifications were relatively low, though the model tended to confuse Brain Tumor cases with Brain Meningioma (64 instances). The classification report further supports these findings, with precision and recall values above 0.9 for Brain Glioma and solid scores for the other classes. Overall, the CNN achieved an accuracy of 8\%, indicating reliable predictive capability, though some overlap between tumor types remains, especially between Brain Tumor and Brain Meningioma.