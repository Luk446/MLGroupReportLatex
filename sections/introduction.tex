\section{Introduction}

In this report, a selection of differing machine learning (ML) techniques are explored. Each technique was chosen based on its area of application. The subject area of medicine was chosen for exploration, and a selection of datasets are explored with different features and formats: (i) a collection of brain magnetic resonance imaging (MRI) images that exhibit three cancer-based pathologies; (ii) a tabular dataset containing patient demographic and healthcare-related information related to stroke pathology; (iii) ?.

Stroke represents a significant global health challenge; it remains one of the most common causes of death and disability \cite{strokestudy}. The application of ML to stroke datasets has become increasingly prevalent as it enables clinicians to identify risks earlier and make more informed treatment decisions \cite{heo_yoon_park_kim_nam_heo_2019}. The objective was to evaluate various ML models to determine their classification performance and assess how imbalanced data impacts prediction accuracy.

The accumulation of valuable medical data has become increasingly common, aiding early detection of health issues and effective treatment \cite{MLS1}. Hospitals are gathering large volumes of structured data, including patients' demographic information, hospital stay details, billing records, and basic health indicators. Such data can also be exploited by machine-learning algorithms to identify patterns that clinicians might not otherwise notice \cite{MLS2}. By analysing these patterns, ML supports clinical decision-making, identifies potential risks, and expedites the diagnostic process \cite{MLS3}. This project evaluates the performance of machine-learning models in predicting medical conditions using structured healthcare data \cite{MLS4}.

To approach the ML problems, an understanding of previous literature on the topic is required for an effective workflow. Academic papers were reviewed to gain an effective skill set and background knowledge required to begin experimentation.
