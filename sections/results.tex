\section{Results}

% ------------ luke -------------------------- %
For the MRI set, the final results of the performance of the model over the testing set can be visualised in the form a confusion matrix, shown in Fig.~\ref{fig:MRICM}. The classification report summarises the model's quantitative performance (see Table~\ref{tab:cnn_classification_report}).

\begin{figure}[h]
    \centering
    \includegraphics[width=0.8\linewidth]{luke_confusion_matrix.png}
    \caption{Confusion matrix for brain cancer set (ReLU)}\label{fig:MRICM}
\end{figure}

\begin{table}[h]
    \centering
    \caption{Classification report for CNN brain tumor prediction (ReLU)}
    \label{tab:cnn_classification_report}
    \begin{tabular}{lcccc}
        \toprule
        \textbf{Class} & \textbf{Precision} & \textbf{Recall} & \textbf{F1-Score} & \textbf{Support} \\
        \midrule
        Brain Glioma   & 0.93 & 0.94 & 0.94 & 301 \\
        Brain Meningioma & 0.77 & 0.91 & 0.84 & 301 \\
        Brain Tumor    & 0.94 & 0.76 & 0.84 & 307 \\
        \midrule
        Accuracy & 0.87 & 0.87 & 0.87 & 0.87 \\
        Macro Avg & 0.88 & 0.87 & 0.87 & 909 \\
        Weighted Avg & 0.88 & 0.87 & 0.87 & 909 \\
        \bottomrule
    \end{tabular}
\end{table}

The result of the differing acitvation function performances are displayed in a line chart showing the running accuracy and loss of the training set and a table of final epoch performance. Shown in, Fig.~\ref{fig:MRITP} (a) and (b), and table \ref{tab:activation_performance}.

\begin{table}[h]
    \centering
    \caption{Final epoch comparison of activation functions for CNN MRI model}\label{tab:activation_performance}
    \begin{tabular}{lcccc}
        \toprule
        \textbf{Activation} & \textbf{Train Loss} & \textbf{Val Loss} & \textbf{Train Acc\%} & \textbf{Val Acc\%} \\
        \midrule
        LReLU & 0.4499 & 0.3561 & 81.67 & 86.47 \\
        SiLU      & 0.4770 & 0.3747 & 80.37 & 85.48 \\
        ReLU      & 0.4588 & 0.4156 & 81.48 & 84.71 \\
        ELU       & 0.5023 & 0.7485 & 79.38 & 63.70 \\
        \bottomrule
    \end{tabular}
\end{table}

\begin{figure}[h]
	\centering
	\begin{subfigure}[b]{0.48\linewidth}
		\includegraphics[width=\linewidth]{activation_train_accuracy_comparison.png}
		\caption{Combined training accuracies for MRI set}\label{fig:CTAMS}
	\end{subfigure}
	\hfil
	\begin{subfigure}[b]{0.48\linewidth}
		\includegraphics[width=\linewidth]{activation_train_loss_comparison.png}
		\caption{Combined training losses for MRI set}\label{fig:CTAMSL}
	\end{subfigure}
	\caption{MRI set training performance}\label{fig:MRITP}
\end{figure}

% ------------ chin -------------------------- %
For the stroke prediction set, the following confusion matrix and classification report summarise the model performance on the test data.
\begin{figure}[h]
    \centering
    \includegraphics[width=0.8\linewidth]{chin_balancedCM.png}
    \caption{Confusion matrix for logistic regression (SMOTE-balanced dataset)}\label{fig:BCM}
\end{figure}

\begin{table}[h]
	\centering
	\caption{Classification report for stroke prediction}
	\label{tab:stroke_classification_report}
	\begin{tabular}{lcccc}
			\toprule
			\textbf{Class} & \textbf{Precision} & \textbf{Recall} & \textbf{F1-Score} & \textbf{Support} \\
		\midrule
		No Stroke & 0.99 & 0.73 & 0.84 & 1458 \\
		Stroke    & 0.13 & 0.79 & 0.22 & 75 \\
		\midrule
		Accuracy  & 0.73 & --   & --   & 1533 \\
		Macro Avg & 0.56 & 0.76 & 0.53 & 1533 \\
		Weighted Avg & 0.94 & 0.73 & 0.81 & 1533 \\
		\bottomrule
	\end{tabular}
\end{table}