\section{Data Analysis and Exploration}

Based on the literature discussed previously, the data used in the following scenarios has undergone careful analysis in order to understand and prepare for upcoming ML.
% ------------ luke -------------------------- %
The MRI classifying dataset was sourced from a comprehensive collection of MRI images collected from a series of hospitals in Bangladesh \cite{rahman_2024}. The data holds great value and significant effort was made to collect and label high quality images.

\subsection{Brain Cancer MRI's}

A sample of the three classes of data present is shown in Fig.~\ref{fig:three_side_by_side}, pictures (a to c). The proprietor has already uniformly resized the images to equal dimensions. The ML for this set uses the hold-out method; 70\% of the data is used for training, with the remaining 30\% divided equally between validation and testing. To avoid bias or overfitting the data is shuffled using an scikit-learn cross-validation function that returns stratified randomised folds \cite{scikit-learn_SS}. For the three sets some further ML specific pre-processing steps are executed, with training having more:

\begin{figure}[h]
    \centering
    \begin{subfigure}[b]{0.3\linewidth}
        \includegraphics[width=\linewidth]{original_brain_glioma.png}
        \caption{Glioma}
        \label{fig:gli}
    \end{subfigure}
    \hfill
    \begin{subfigure}[b]{0.3\linewidth}
        \includegraphics[width=\linewidth]{original_brain_menin.png}
        \caption{Meningioma}
        \label{fig:menin}
    \end{subfigure}
    \hfill
    \begin{subfigure}[b]{0.3\linewidth}
        \includegraphics[width=\linewidth]{original_brain_tumor.png}
        \caption{Tumor}
        \label{fig:tumor}
    \end{subfigure}
    \caption{Sample images from MRI scans}
    \label{fig:three_side_by_side}
\end{figure}

\begin{figure}[h]
    \centering
    \begin{subfigure}[b]{0.3\linewidth}
        \includegraphics[width=\linewidth]{preprocessed_brain_glioma.png}
        \caption{Glioma}
        \label{fig:pregli}
    \end{subfigure}
    \hfill
    \begin{subfigure}[b]{0.3\linewidth}
        \includegraphics[width=\linewidth]{preprocessed_brain_menin.png}
        \caption{Meningioma}
        \label{fig:premenin}
    \end{subfigure}
    \hfill
    \begin{subfigure}[b]{0.3\linewidth}
        \includegraphics[width=\linewidth]{preprocessed_brain_tumor.png}
        \caption{Tumor}
        \label{fig:pretumor}
    \end{subfigure}
    \caption{Preprocessed sample images from MRI scans}
    \label{fig:pre_three_side_by_side}
\end{figure}

\begin{itemize}
    \item Random resized crop to 224$\times$224 pixels
    \item Random horizontal flip with 50\% probability
    \item Conversion to 3-channel greyscale
    \item Conversion to tensor format
    \item Normalisation to floating-point values in range [0,1]
    \item Standardisation using ImageNet mean and standard deviation
\end{itemize}

% ------------ chin -------------------------- %
Additionally, the stroke dataset presented some inherent issues.

\subsection{Stroke patient data}

The stroke dataset used contains 5,110 patient records with demographic and clinical attributes such as age, heart disease, BMI, hypertension and average blood sugar level. The main variable (stroke) is binary. A small amount of missing BMI data was filled in by using the mean value.

A major challenge with this dataset is its severe imbalance, with only approximately 3.4\% of cases representing actual stroke occurrences. The data exploration indicated a higher probability of stroke among individuals within the age range of 50--80 years, as shown in Fig.~\ref{fig:ADSO}, and those with elevated glucose levels. When correlation analysis was carried out, it showed that age, hypertension and heart disease had subtle but meaningful relationship with stroke occurrences.

\begin{figure}[h]
    \centering
    \includegraphics[width=0.8\linewidth]{chin_agedistgraph.png}
    \caption{Age Distribution by Stroke Outcome}\label{fig:ADSO}
\end{figure}

To get the data ready for training, word-based features were converted to 0/1 columns so the models could process them. All numerical values were also rescaled to a similar range to ensure no feature dominated the other while the model was trained. To deal with the class imbalance problem, SMOTE was used to generate synthetic stroke cases, which helped in detecting rare instances of stroke. The preprocessing and exploration guided the modelling choices in later stages.

After the initial data-processing step has been completed for all the datasets, the ML algorithms can now be established.

% ---------------- sar+jah ------------- %
\subsection{Combined Healthcare Dataset}

This study also considers a combined healthcare dataset assembled for multiclass disease prediction using non-clinical attributes and symptom indicators. We treat the administrative/demographic records and symptom-based records jointly as a single, combined dataset for analysis and modelling.

Within the combined dataset, the symptom-based component is concerned with general disease prediction using data on symptoms. It is a large array of symptom features, whereas the last column is the diagnosed disease. The problem is represented as a multiclass classification task in which the model will be required to estimate one category of disease through a combination of symptoms. The dataset is used to train machine-learning algorithms like the Random Forest classifier to learn the relationship between patterns of symptoms and the occurrence of certain diseases and to assess the ability of symptoms alone to aid in the accurate prediction of a disease \cite{MLS3,MLS4}.

\subsubsection{Healthcare multiclass dataset}

The healthcare multiclass portion includes 55{,}500 patient records, each labelled with one of six medical conditions: Arthritis, Asthma, Cancer, Obesity, Hypertension, or Diabetes. Each condition accounts for roughly 16--17\% of the entire number of samples and, therefore, the dataset is perfectly balanced. To explore the range and preparedness of these demographic and administrative properties, we explored variables including age, sex, blood type, insurance provider, billing amount, and length of stay to assess variability. No empty or duplicate entries were noticed and verification that the dataset is clean and complete was done. The pie chart in Fig.~\ref{fig:pieP} shows the even spread among the six medical conditions, to help make sure that when the model learns from the data, it should not favour any one class. Now this dataset is non-clinical, but it does provide a good foundation to explore how well the demographic and administrative attributes support multiclass disease prediction \cite{MLS1,MLS2}.

\begin{figure}
    \centering
    \includegraphics[width=0.8\linewidth]{piechart.png}
    \caption{Patient distribution pie chart}\label{fig:pieP}
\end{figure}

\subsubsection{Symptom-based disease prediction}

The symptom disease dataset is a pool of patient records of symptoms, and the last column is the disease that the patient has. The data features several binary symptom attributes that represent the absence or presence of a given symptom and one target attribute that represents the disease that was diagnosed. It was verified that all the entries are complete and there are no missed or repeated values, which proves the data is clean and can be used for training. To get familiar with the data structure, the distribution of diseases over the dataset was analysed. All the disease categories are adequately represented and, therefore, the model will get balanced samples during training and will not be biased in any way to one type of classification. The variability of the symptom features is also good and this enables the classifier to learn various symptom patterns of varying diseases. Even though there are no clinical measurements or numerical health indicators available in the dataset, the large variety of symptoms gives a solid foundation on multiclass disease prediction \cite{MLS3}. This makes the dataset appropriate for training machine-learning models, e.g., the Random Forest classifier \cite{MLS4}.